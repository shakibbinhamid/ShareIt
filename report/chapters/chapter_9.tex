\chapter{Conclusion}
\label{chap:conclusion}

Even though \textit{Software Engineering} itself is now considered an established and mature study, application development for smartphones is a recent and diverse addition in the industry. Because of the wide variety of smartphones used by the people, there is also a plethora of development toolkits available for multiple different platforms. Although each major mobile OS has its own distinct engineering practices and architectural patterns, it is expected that an app is to be implemented and supported well simultaneously for multiple platforms. This project has tried to tackle the problem by using hybrid technologies to develop a mobile app.\\

It is also extremely difficult to rigidly define mobile app requirements in this volatile industry and satisfy user experiences for multiple platforms at the same time. I have successfully used scrum (iterative development and review) to develop the app, and organised interview sessions to engage the end-users in requirements elicitation, design and evaluate the app in hopes to evade such difficulties.\\

However, the benefits of fast development and multi-platform support comes with the price of automated testing on client side. While there are established practices in server building using NodeJS, apps built with such hybrid client frameworks as Ionic are nearly impossible to unit test because of lack of mocking and isolation support. In fact, there seems to be very few, if any, example of community support in this regard. This is certainly a barrier, but given the aims of the project, the issue can be considered a discovery of fact, rather than a major shortcoming.\\

Finally, I consider the project to be successful, and valuable for future students, because it demonstrates relevant agile software development processes such as scrum, daily testing, evolving user stories etc. within the short project frame, experiments with emergent technologies in mobile app industry and evaluates the results against end users' expectations.