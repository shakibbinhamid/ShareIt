\chapter{Introduction}

Technology has introduced new trading concepts that can expand a traditional sharing community far beyond that of close friends. Take eBay for example - an eBay user can post the items that they want to sell or put on auction,  with name, description and pictures to help a buyer minimise the risks in transactions. A buyer can also find ratings-reviews about the seller's previous transactions etc. Both buyer and seller can securely pay via Paypal with buyer protection. These technology platforms themselves are not providing any items; they are not product brands. Rather, they provide some features that build trust in their communities, because sharing, swapping, renting etc. only happens when there is trust between the participating parties. Such Sharing Economy using technology is called \textit{Collaborative Consumption} (CC).\\

University of Southampton (UoS) itself has a vibrant student community. Students here borrow calculators and books for the odd coursework, parking spots for their visiting parents, sell or give away their old mobiles, music players or laptops etc. This happens often enough that there are several social media groups for sharing items. A targeted Collaborative Consumption tool for UoS students can certainly thrive because of the existing communities. This tool needs to be mobile because an average student usually accesses the aforementioned social media sites from their phones, on the go.\\

However, the goal of this project is not promoting CC, rather it is to explore modern software engineering practices - 

\begin{description}
	\item [scrum] : develop in short, targeted and self contained cycles,
	\item [participatory design] : include end users in requirements elicitation and user experience testing,
	\item [user centred development] : develop towards user stories rather than use cases,
	\item [behaviour driven development] : write automated test scripts for each user story under preconditions.
\end{description}

Another goal is to experience both \textit{server} and \textit{client} side development to demonstrate how mobile application features such as push notifications etc. operate. The premise of a mobile CC tool is suitable to demonstrate such processes.\\

Mobile applications for smartphones usually target a wide range of users from very different backgrounds. So, evaluating User Experience (UX) for these applications is essential and time should be spent on user participations. However, development in the native language for a platform, Java on Android for example, usually requires a very steep learning curve. Together with the need to develop a full stack application in such a short time, a fast mobile development framework is required. Such tools exist that it is possible to build mobile applications entirely in JavaScript (both server and client) and deploy the same code to multiple platforms. As this project's resultant application is a proof-of-concept rather than a marketplace application, it is quite acceptable and even preferred to work with such \textit{Hybrid} solutions.\\

Overall, the goal of this project is to demonstrate modern Software Engineering practices by implementing a Client Hybrid Mobile Collaborative Application and the corresponding Server Application.\\

Below is a summary of what content is to follow -

\begin{description}
	\item [Chapter 2] reviews the relevant literature on collaborative consumption, hybrid mobile development and agile software development practices.
	\item [Chapter 3] lists the features of the client app and discusses what features were discarded and why.
	\item [Chapter 4] explains how the project was organised/structured and managed during different stages such as sprints etc.
	\item [Chapter 5] gives an overview of the technologies used and discusses the architecture of the system as a whole.
	\item [Chapter 6] explains the client and server implementation in-depth.
	\item [Chapter 7] discusses how the implementation was tested - automated and against scenarios.
	\item [Chapter 8] gives a detailed description of why and how the user sessions were organised, performed and documented.
	\item [Chapter 9] evaluates the success, shortcomings and usefulness of the project.
	\item [Chapter 10] concludes the project aims and results.
\end{description}